\begin{table}\renewcommand{\arraystretch}{1.1}
  %\centering
  \caption{I denne tabellen er det vist de geometriske størrelsene til jernklumpene, parallellt med rotasjonsakse ($a_{\parallel}$) og tangensialt på rotasjonsaksen ($a_{\perp}$). Fra disse måline kan en beregne avmagnetiseringsfaktoren parallellt med magnetfeltet ($D_{\parallel}$) og tangensialt på magnetfeltet ($D_{\perp}$).}
  \label{faraday}
  \begin{tabular}{|l|S|S|S|S|S|S|}
    \colrule
      Bølgelengde $\lambda$ &
      \multicolumn{2}{|l|}{$440$nm\qquad\qquad} &
      \multicolumn{2}{l|}{$580$nm\qquad\qquad} &
      \multicolumn{2}{l|}{$595$nm\qquad\qquad} \\
      \colrule

      $B$ [mT] & \multicolumn{6}{c|}{Vinkel $\theta\pm0.2$ [$\degree$] } \\   \colrule

      43(1)  & 1.8 & 1.4 & 1.8 & 1.4 & 2.0 & 1.6 \\
      63(1)  & 2.8 & 2.2 & 2.6 & 2.0 & 2.8 & 2.6 \\
      83(2)  & 3.8 & 3.0 & 3.0 & 2.8 & 3.6 & 3.2 \\
      102(2) & 4.2 & 4.0 & 4.0 & 3.4 & 4.4 & 4.2 \\
      119(3) & 5.2 & 4.8 & 4.4 & 4.0 & 5.2 & 4.8 \\ \colrule

      Strømretning & {$I_+$} & {$I_-$} & {$I_+$} & {$I_-$} & {$I_+$} & {$I_-$} \\ \colrule
      Lineærregresjon &
      \multicolumn{2}{|c}{$A+B$} &
      \multicolumn{2}{|c}{$C+D$} &
      \multicolumn{2}{|c|}{$E+F$} \\
      \colrule
  \end{tabular}
\end{table}
