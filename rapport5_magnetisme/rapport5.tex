% ****** Start of file apssamp.tex ******
%
%   This file is part of the APS files in the REVTeX 4.1 distribution.
%   Version 4.1r of REVTeX, August 2010
%
%   Copyright (c) 2009, 2010 The American Physical Society.
%
%   See the REVTeX 4 README file for restrictions and more information.
%
% TeX'ing this file requires that you have AMS-LaTeX 2.0 installed
% as well as the rest of the prerequisites for REVTeX 4.1
%
% See the REVTeX 4 README file
% It also requires running BibTeX. The commands are as follows:
%
%  1)  latex apssamp.tex
%  2)  bibtex apssamp
%  3)  latex apssamp.tex
%  4)  latex apssamp.tex
%
\documentclass[%
 reprint,
%superscriptaddress,
%groupedaddress,
%unsortedaddress,
%runinaddress,
%frontmatterverbose,
%preprint,
%showpacs,preprintnumbers,
%nofootinbib,
%nobibnotes,
%bibnotes,
 amsmath,amssymb,
 aps,
%pra,
%prb,
%rmp,
%prstab,
%prstper,
%floatfix,
]{revtex4-1}

\usepackage[utf8]{inputenc}
\usepackage[norsk]{babel}
\usepackage{varioref}
\usepackage{graphicx}% Include figure files
\usepackage{enumitem}
\usepackage{dcolumn}% Align table columns on decimal point
\usepackage{bm}% bold math
\usepackage[margin=0.9in]{geometry}
\usepackage[mathlines]{lineno}% Enable numbering of text and display math
%\linenumbers\relax % Commence numbering lines

\usepackage[usenames,dvipsnames,svgnames,table]{xcolor}
\usepackage[colorlinks]{hyperref}
\usepackage{relsize}
%\usepackage{booktabs}
\usepackage{graphicx,verbatim,amsfonts,geometry}
\usepackage{amsmath}
\newcommand*\diff{\mathop{}\!\mathrm{d}}
\newcommand*\Diff[1]{\mathop{}\!\mathrm{d^#1}}
\usepackage{ulem}
\usepackage{amssymb}
\usepackage{multirow}
\usepackage{soul}
\usepackage{dsfont}
% allows for temporary adjustment of side margins
\usepackage{chngpage}
% just makes the table prettier (see \toprule, \bottomrule, etc. commands below)
\usepackage{booktabs}

\usepackage{commath}
\usepackage{wrapfig}
\usepackage[free-standing-units=true]{siunitx}
\DeclareSIUnit\year{yr}
\usepackage{gensymb}
\newcommand{\ROM}[1]{%
  \textup{\uppercase\expandafter{\romannumeral#1}}%
}
\usepackage{physics}
\usepackage{caption}
\usepackage{bm}
\usepackage{gensymb}
%\usepackage[showframe,%Uncomment any one of the following lines to test
%%scale=0.7, marginratio={1:1, 2:3}, ignoreall,% default settings
%%text={7in,10in},centering,
%%margin=1.5in,
%%total={6.5in,8.75in}, top=1.2in, left=0.9in, includefoot,
%%height=10in,a5paper,hmargin={3cm,0.8in},
%]{geometry}

\begin{document}

%\preprint{APS/123-QED}

\title{Magnetisme}% Force line breaks with \\

\author{Ivar Svalheim Haugerud}

\affiliation{%
 Universitetet i Oslo\\
}%

\date{\today}% It is always \today, today,
             %  but any date may be explicitly specified

\begin{abstract}
Abstract. Magnetisme er kult, Skår er tøff.
\end{abstract}

\pacs{Valid PACS appear here}% PACS, the Physics and Astronomy
                             % Classification Scheme.
%\keywords{Suggested keywords}%Use showkeys class option if keyword
                              %display desired
\maketitle

%\tableofcontents

\section{\label{sec:level1}Introduksjon}
\section{\label{sec:level2}Teori}
Hele den klassiske elektromagnetisme kan beskrives ved hjelp av fire partsielle differensiallikninger. Disse likningene beskriver elektriske $\bm{E}$ og magnetiske $\bm{B}$ felt, og forklarer Lorentz kraften, klassisk optikk og elektriske kretser. Likningene kan skrives på flere måter, og på flere former. I denne rapporten velger vi å bruke følgende:
\begin{align}
  \nabla \cdot \bm{E} &= \frac{\rho}{\epsilon_0} \\
  \nabla \cdot \bm{B} &= 0 \\
  \nabla \cross \bm{E} &= -\pdv{\bm{B}}{t} \\
  \nabla \cross \bm{B} &= \mu_0\left(\bm{J} + \epsilon_0\pdv{\bm{E}}{t}\right)
\end{align}
I likningene er $\rho$ den elektriske tettheten, og $\epsilon$ er permittiviteten, som beskriver motstanden et medie har mot et påtrykt elektrisk felt. I disse likningene brukes $\epsilon_0$, som er en naturkonstant, som angir permittiviteten i vakum. $\mu$ er et mål på matrialers evne til å magnetiseres av et ytre påtrykket magnetfelt. I disse likningene brukes vakuumpermeabiliteten $\mu_0$ som er en naturkonstant, som angir permeabiliteten i vakum. Permeabitilteten i et materialet kan skrives som produktet av vakuumpermeabiliteten, og den relative permeabiliteten $\mu_r$. Fra dette kan en få den dimensjonsløse egenskapen magnetiske susceptibiliteten $\chi = \mu_r - 1$. Den magnetiske susceptibiliteten forteller oss om materialet er tiltrukket, eller frastøtet, av materialet. Den magnetiske susceptibiliteten beskrives hva slags magnetisk materialet vi ser på. Superledere er perfekte diamagneter, de setter opp et magnetfelt som eksakt kanselerer et påtrykt magnetfelt, $\chi=-1$. $\bm{J}$ beskriver strømtetthet, det vil si elektrisk strøm $I$ gjennom et flateareal $A$. \\
Maxwell's likninger kan også skrives ved $\bm{H}$-feltet istedenfor $\bm{B}$-feltet. Hvor $\bm{B}$-feltet representer den totale magnetiske flukstettheten, fra alle kilder. Definisjonen av $\bm{H}$-feltet er
\begin{equation}
  \bm{H} = \frac{\bm{B}}{\mu_0}-\bm{M},
\end{equation}
hvor $\bm{M}$, er magnetiseringen av materialet. Hvor $\bm{H}$ kalles den magnetiske feltstyrken, og representerer magnetfeltet som ikke kommer av magnetiseringen til materialet $\bm{M}$. Ved å bruke $\bm{H}$-felt istedenfor $\bm{B}$-felt i Maxwell's likninger kan en skrive om Amperes lov, på differensialform, hvis vi antar konstant elektrisk felt og ingen fri strøm, til å være
\begin{equation}
  \nable \cross \bm{H} = 0.
\end{equation}På samme måte kan en skrive om Gauss' lov, på differensialform, til å være
\begin{equation}
  \nabla \cdot \left(\bm{H}+\bm{M}\right) = 0.
\end{equation}
\\
Maxwell's likninger kobler sammen elektriske og magnetiske felt. Magnetiske felt stammer fra ladde partikkler i bevegelse. Elektronet har også en egenspinn, som gjør at eleketronet kan sees på som et dipol. I atomer beveger elektroner seg i bane rundt atomet, og danner et magnetfelt fra angulærmomentet. Hvordan angulærmomentet til de forskjllige elekronene rundt materialet er satt sammen, avgjør de magnetiske egenskapene til materialet atomet består av. Dette resulterer i forskjellige typer magnetiske materialer: diamagnetiske, paramagnetiske og ferromagnetiske.
\subsection{Diamagnetisme}
De aller fleste materialer er ikke magnetiske. Slike materialer kaller vi for diamagnetiske materialer, som klasifiseres ved at de ikke er magnetiske, uten en ytre påvirkning. Materialet kan bli magnetisert ved å påtrykke et ytre magnetfelt som vil elektronorbitalene deformeres. Ved Lenz regel vil dette motsette seg forandringen av feltstyrken. Dette betyr at diamagnetiske materialer danner et magnetfelt som motsetter seg det ytre påtrykte magnetfeltet. Dette betyr at den magnetiske susceptibiliteten til materialet er negativ $\chi < 0$. Lenz regel gjelder for alle atomer, og følgelig alle materialer, men denne effekten er svært liten iforhold til andre magnetiske effekter. Selv om den er liten er den viktig siden den påvirker alle materialer. Det kan vises, med utgangspunkt i potensiell energi for en magnetisk dipol, med dipolmoment $\mu$, i et magnetfelt, at den magnetiske kraften som virker er gitt av
\begin{equation}
  F_Z = -\frac{\chi}{2\mu_0}A\left(B_1^2 - B_2^2\right).
\end{equation}
Hvor $B_1$ og $B_2$ er magnetfeltet på tvers av symetriaksen til staven, i henholdsvis bunn og toppen av staven, og $A$ er tversnittsarealet.
\subsection{Paramagnetisme og ferromagnetisme}
I diamagnetiske materialer kanselerte spinnet til elektronene slik at netto angulært moment er tilnærmet null. For andre materialer, hvor spinnet til elektronene ikke kanselerer hverandre, vil det bevegelsen til elektronene danne et netto magnetisk moment $\mu$. Er den gjennomsnittlige orienteringen til atomenes magnetfelt i en bestemt retning, vil summen av alle magnetfeltene danne et magnetisk felt som kan merkes utenfor mediet. Dette betyr at materialet er \textit{magnetisert}, som beskrives med paramteren $\bm{M} = \diff\bm{\mu}/\diff V$, hvor $\bm{\mu}$ er det magnetiske moment, og $V$ er et volumelement. Materialer som oppfører seg slik klasseifiseres som paramagnetiske og ferromagnetiske materialer.\par
Paramagnetiske materialer vil ikke kunne danne et magnetfelt alene, og får det kun fra et ytre påtrykt magnetisk felt. Det ytre påtrykte magnetfelt vil få alle de magnetiske dipolene til å rette seg inn samme retning. Fjernes feltet vil orienteringen bli tilfeldig igjen, og materialet vil ikke lenger danne et magnetisk felt. Magnetiseringen av materialet er omtrent proposjonalt med styrken på det ytre påtrykte magnetfeltet $\bm{H}$, $\bm{M} = \chi\bm{H}$, hvor den magnetiske susceptibiliteten $0<\chi\ll1$. En paramagnet forsterker det påtrykte magnetfeltet.
\par
Ferromagnetiske materialer oppfører seg som paramagnetiske materialer, men størrelsen på magnetiseringen er mye større. Forsterkningen av det ytre påtrykte magnetfeltet kan være opp mot en faktor $10 000$ sterkere \cite{oppgave}. Den sterke magnetiseringen fører til at de atomære dipolene klarer å oppretholde magnetfeltet sitt, etter at det ytre magnetfeltet er fjernet. Dette betyr at man kan lage en permanentmagnet ved å magnetisere et ferromagnetisk materiale. For ferromagneter er den magnetiske susceptibiliteten ikke en materialkonstant. Det er ikke et lineært forhold mellom det ytre påtrykte magnetfeltet og den magnetiske susceptibiliteten, susceptibiliteten er avhengig av geometrien til materialet, og styrken på det påtrykte magnetfeltet.
\subsection{Avmagnetisering}
For å avmagnetisere et ferromagnetisk materiale tregns det et magnetisk felt, styrken på mangetfeltet som trengs kalles avmagnetiseringsfeltet $\bm{H}_d$, er gitt av
\begin{equation}
  H_{i, d} = D_iM_i,
\end{equation}
for ellipsoider med uniform magnetisering. I likningen representerer indeksen $i$ en retning $(x, y, z)$, og $D_i$ er avmagnetiseringsfaktoren. Avmagnetiseringsfaktoren kan beregnes analytisk fra å vite formen på ellipsoiden. Formen på ellipsoiden kan klassifiseres ved ett tall, eksentrisiteten $\epsilon$, som er gitt av
\begin{equation}
  \epsilon = \sqrt{1-\frac{1}{f^2}},
\end{equation}
hvor $f$ er gitt av
\begin{equation}
  f = \frac{a_{\parallel}}{a_{\perp}}.
\end{equation}
I likningene er $a_{\parallel}$ lengden parallellt med rotasjonsaksen til ellipsoiden, og $a_{\perp}$ er lengden til ellipsoiden ortogonalt på rotasjonsaksen. Fra å vite eksentrisiteten til ellipsoiden kan en beregne avmagnetiseringsfaktoren parallelt med $D_{\parallel}$, og ortogonalt på $D_{\perp}$, rotasjonsaksen
\begin{align}
  D_{\parallel} &= \left(1-\frac{1}{\epsilon^2}\right)\left(1-\frac{1}{2\epsilon}\ln{\left(\frac{1+\epsilon}{1-\epsilon}\right)} \right) \\
  D_{\perp} &= \frac{1-D_{\parallel}}{2}
\end{align}
%Dette kommer blant annet av at elektroner foretrekker å motsatt spinn, etter Hund's regel.

\section{\label{sec:level2}Eksperimentet}
\section{\label{sec:level2}Resultater}



\begin{thebibliography}{9}


\bibitem{squires}
Squires, G.L. \emph{Practical Physics}, Cambridge University Press, 2001.



 \end{thebibliography}

\end{document}
%
% ****** End of file apssamp.tex ******
